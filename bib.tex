
@article{devogel_evaluation_2017,
	title = {Evaluation of predictive performance of injury risk curves},
	journal = {54th Annual Rocky Mountain Bioengineering Symposium},
	author = {DeVogel, Nicholas and Yoganandan, Narayan and Pintar, Frank and Banerjee, Anjishnu},
	month = apr,
	year = {2017}
}

@article{szabo_partial_2017,
	title = {Partial observability with exchangeable binary outcomes},
	journal = {Joint Statistical Meetings},
	author = {Szabo, Aniko and DeVogel, Nicholas},
	month = aug,
	year = {2017}
}

@article{szabo_partial_2017-1,
	title = {Partial observability with exchangeable binary outcomes},
	journal = {BiostatMCW 2017},
	author = {Szabo, Aniko and DeVogel, Nicholas},
	month = aug,
	year = {2017}
}

@article{devogel_assessment_2018,
	title = {Assessment of covariate adjustment methods for binary outcomes in clinical trials},
	journal = {FDA ORISE Conference},
	author = {DeVogel, Nicholas and Bein, Ed},
	month = aug,
	year = {2018}
}

@article{devogel_injury_2019,
	title = {Injury {Risk} {Curves} {Using} a {Novel} ({Bayesian}) {Technique} to {Describe} {Human} {Tolerance} in {Impact} {Biomechanics}},
	journal = {Summer Biomechanics, Bioengineering and Biotransport Conference (Abstract accepted)},
	author = {DeVogel, Nicholas and Banerjee, Anjishnu and Yoganandan, Narayan},
	month = jun,
	year = {2019}
}

@article{razmjoo_normative_2018,
	title = {Normative {Study} of the {Intervertebral} {Disk} {Measurement} in {Young} {Healthy} {Adults}.},
	journal = {Spine and Peripheral Nerves Annual Meeting},
	author = {Razmjoo, Shabani and DeVogel, Nicholas and Yoganandan, Narayan and Baisden, Jamie},
	month = mar,
	year = {2018}
}

@article{devogel_comparison_2019,
	title = {Comparison of {Hypothesis} {Testing} {Methods} on {Random} {Genetic} {Effects} in {Family} {Data}},
	journal = {Eastern North American Region Conference},
	author = {DeVogel, Nicholas and Wang, Tao},
	month = mar,
	year = {2019}
}

@article{devogel_assessment_2018-1,
	title = {Assessment of population sub-structure in genetic association studies.},
	journal = {Population, Evolutionary and Quantitative Genetics Conference},
	author = {DeVogel, Nicholas and Wang, Tao},
	month = may,
	year = {2018}
}

@article{devogel_novel_2019,
	title = {A novel analytical tool to assess spine injury risk in impact biomechanics},
	journal = {15th Annual Injury Biomechanics Symposium (Abstract accepted)},
	author = {DeVogel, Nicholas and Yoganandan, Narayan and Pintar, Frank A. and Banerjee, Anjishnu},
	month = may,
	year = {2019},
	note = {ABSTRACT ACCEPTED}
}

@article{khan_outcomes_2015,
	title = {Outcomes of {Methicillin} {Sensitive} {Staphylococcus} aureus ({MSSA}) {Pneumonia} in {Adult} {Patients} ({\textgreater}18 {Years})},
	volume = {148},
	issn = {0012-3692},
	url = {https://journal.chestnet.org/article/S0012-3692(16)36035-4/abstract},
	doi = {10.1378/chest.2277202},
	abstract = {SESSION TITLE: Chest infections Posters II},
	language = {English},
	number = {4},
	urldate = {2019-03-12},
	journal = {CHEST},
	author = {Khan, Ezza and Baig, Saqib and Szabo, Aniko and DeVogel, Nicholas},
	month = oct,
	year = {2015},
	pages = {126A}
}

@article{baig_outcomes_2015,
	title = {Outcomes of {Methicillin} {Resistant} {Staphylococcus} aureus ({MRSA}) {Pneumonia} in {Adult} {Patients} ({\textgreater}18 {Years})},
	volume = {148},
	issn = {0012-3692},
	url = {https://journal.chestnet.org/article/S0012-3692(16)36036-6/abstract},
	doi = {10.1378/chest.2274855},
	abstract = {SESSION TITLE: Chest infections Posters II},
	language = {English},
	number = {4},
	urldate = {2019-03-12},
	journal = {CHEST},
	author = {Baig, Saqib and Khan, Ezza and Szabo, Aniko and DeVogel, Nicholas},
	month = oct,
	year = {2015},
	pages = {127A}
}

@article{baig_risk_2016,
	series = {American {Thoracic} {Society} {International} {Conference} {Abstracts}},
	title = {Risk {Factors} for the {Development} of {Acute} {Respiratory} {Distress} {Syndrome} ({ARDS}) in {Patients} with {Pneumonia}, a {Nationwide} {Retrospective} {Study} {Using} the {Nationwide} {Inpatient} {Sample}  ({NIS}) {Database} from {Year} 2002-2012},
	url = {https://www.atsjournals.org/doi/abs/10.1164/ajrccm-conference.2016.193.1_MeetingAbstracts.A1826},
	urldate = {2019-03-12},
	journal = {A53. RESPIRATORY FAILURE: RISK FACTORS AND OUTCOMES IN ARDS},
	author = {Baig, Saqib and Ahmad, Shahryar and Devogel, Nicholas and Szabo, Aniko and Thandra, Krishna and Hussain, Jawad and Khan, Ezza},
	month = may,
	year = {2016},
	doi = {10.1164/ajrccm-conference.2016.193.1_MeetingAbstracts.A1826},
	pages = {A1826--A1826}
}

@article{devogel_ranking_2018,
	title = {Ranking of {Biomechanical} {Metrics} to {Describe} {Human} {Response} to {Impact}-{Induced} {Damage}},
	url = {http://dx.doi.org/10.1115/IMECE2018-88007},
	doi = {10.1115/IMECE2018-88007},
	abstract = {Determination of human tolerance to impact-induced damage or injury is needed to assess and improve safety in military, automotive, and sport environments. Impact biomechanics experiments using post mortem human surrogates (PMHS) are routinely used to this objective. Risk curves representing the damage of the tested components of the PMHS are developed using the metrics gathered from the experimental process. To determine the metric that best explains the underlying response to the observed damage, statistical analysis is required of all the output response metrics (such as peak force to injury) along with the examination of potential covariates. This is conducted by parametric survival analysis. The objective of this study is to present a robust statistical methodology that can be effectively used to achieve these goals by choosing the best metric explaining injury and provide a ranking of the metrics. Previously published data from foot-ankle-lower leg experiments were used with two possible forms of censoring: right and left censoring or right and exact censoring, representing the no injury and injury data points in a different manner. The statistical process and scoring scheme were based on the predictive ability assessed by the Brier Score Metric (BSM) which was used to rank the metrics. Response metrics were force, time to peak, and rate. The analysis showed that BSM is effective in incorporating different covariates: age, posture, stature, device used to deliver the impact load, and the personal protective equipment (PPE), i.e., military boot. The BSM-based analysis indicated that the peak force was the highest ranked metric for the exact censoring scheme and the age was a significant covariate, and that peak force was also the highest ranked metric for the left censored scheme and the PPE covariate was statistically significant. IRCs are presented for the best metric.},
	urldate = {2019-03-12},
	author = {DeVogel, Nicholas and Banerjee, Anjishnu and Pintar, Frank A. and Yoganandan, Narayan},
	month = nov,
	year = {2018},
	pages = {V003T04A064}
}

@article{yoganandan_foot-ankle_2017,
	title = {Foot-ankle complex injury risk curves using calcaneus bone mineral density data},
	volume = {72},
	issn = {1878-0180},
	doi = {10.1016/j.jmbbm.2017.05.010},
	abstract = {OBJECTIVE: Biomechanical data from post mortem human subject (PMHS) experiments are used to derive human injury probability curves and develop injury criteria. This process has been used in previous and current automotive crashworthiness studies, Federal safety standards, and dummy design and development. Human bone strength decreases as the individuals reach their elderly age. Injury risk curves using the primary predictor variable (e.g., force) should therefore account for such strength reduction when the test data are collected from PMHS specimens of different ages (age at the time of death). This demographic variable is meant to be a surrogate for fracture, often representing bone strength as other parameters have not been routinely gathered in previous experiments. However, bone mineral densities (BMD) can be gathered from tested specimens (presented in this manuscript). The objective of this study is to investigate different approaches of accounting for BMD in the development of human injury risk curves.
METHODS: Using simulated underbody blast (UBB) loading experiments conducted with the PMHS lower leg-foot-ankle complexes, a comparison is made between the two methods: treating BMD as a covariate and pre-scaling test data based on BMD. Twelve PMHS lower leg-foot-ankle specimens were subjected to UBB loads. Calcaneus BMD was obtained from quantitative computed tomography (QCT) images. Fracture forces were recorded using a load cell. They were treated as uncensored data in the survival analysis model which used the Weibull distribution in both methods. The width of the normalized confidence interval (NCIS) was obtained using the mean and ± 95\% confidence limit curves.
PRINCIPAL RESULTS: The mean peak forces of 3.9kN and 8.6kN were associated with the 5\% and 50\% probability of injury for the covariate method of deriving the risk curve for the reference age of 45 years. The mean forces of 5.4 kN and 9.2kN were associated with the 5\% and 50\% probability of injury for the pre-scaled method. The NCIS magnitudes were greater in the covariate-based risk curves (0.52-1.00) than in the risk curves based on the pre-scaled method (0.24-0.66). The pre-scaling method resulted in a generally greater injury force and a tighter injury risk curve confidence interval. Although not directly applicable to the foot-ankle fractures, when compared with the use of spine BMD from QCT scans to pre-scale the force, the calcaneus BMD scaled data produced greater force at the same risk level in general.
CONCLUSIONS: Pre-scaling the force data using BMD is an alternate, and likely a more accurate, method instead of using covariate to account for the age-related bone strength change in deriving risk curves from biomechanical experiments using PMHS. Because of the proximity of the calcaneus bone to the impacting load, it is suggested to use and determine the BMD of the foot-ankle bone in future UBB and other loading conditions to derive human injury probability curves for the foot-ankle complex.},
	language = {eng},
	journal = {Journal of the Mechanical Behavior of Biomedical Materials},
	author = {Yoganandan, Narayan and Chirvi, Sajal and Voo, Liming and DeVogel, Nicholas and Pintar, Frank A. and Banerjee, Anjishnu},
	year = {2017},
	pmid = {28505593},
	keywords = {Accidents, Traffic, Ankle Injuries, Bone Density, Cadaver, Calcaneus, Foot Injuries, Fracture, Humans, Impact biomechanics, Lower leg, Peak force, Postmortem human subjects},
	pages = {246--251}
}

@article{yoganandan_role_2018,
	title = {Role of disc area and trabecular bone density on lumbar spinal column fracture risk curves under vertical impact},
	volume = {72},
	issn = {1873-2380},
	doi = {10.1016/j.jbiomech.2018.02.030},
	abstract = {While studies have been conducted using human cadaver lumbar spines to understand injury biomechanics in terms of stability/energy to fracture, and physiological responses under pure-moment/follower loads, data are sparse for inferior-to-superior impacts. Injuries occur under this mode from underbody blasts.
OBJECTIVES: determine role of age, disc area, and trabecular bone density on tolerances/risk curves under vertical loading from a controlled group of specimens. T12-S1 columns were obtained, pretest X-rays and CTs taken, load cells attached to both ends, impacts applied at S1-end using custom vertical accelerator device, and posttest X-ray, CT, and dissections done. BMD of L2-L4 vertebrae were obtained from QCT. Survival analysis-based Human Injury Probability Curves (HIPCs) were derived using proximal and distal forces. Age, area, and BMD were covariates. Forces were considered uncensored, representing the load carrying capacity. The Akaike Information Criterion was used to determine optimal distributions. The mean forces, ±95\% confidence intervals, and Normalized Confidence Interval Size (NCIS) were computed. The Lognormal distribution was the optimal function for both forces. Age, area, and BMD were not significant (p {\textgreater} 0.05) covariates for distal forces, while only BMD was significant for proximal forces. The NCIS was the lowest for force-BMD covariate HIPC. The HIPCs for both genders at 35 and 45 years were based on population BMDs. These HIPCs serve as human tolerance criteria for automotive, military, and other applications. In this controlled group of samples, BMD is a better predictor-covariate that characterizes lumbar column injury under inferior-to-superior impacts.},
	language = {eng},
	journal = {Journal of Biomechanics},
	author = {Yoganandan, Narayan and Moore, Jason and Pintar, Frank A. and Banerjee, Anjishnu and DeVogel, Nicholas and Zhang, JiangYue},
	year = {2018},
	pmid = {29559244},
	keywords = {Adult, Aged, Biomechanical experiments, Bone Density, Cadaver, Cancellous Bone, Confidence intervals, Humans, Impact loading, Lumbar Vertebrae, Lumbar spine, Male, Middle Aged, Probability, Probability curves, Radiography, Risk, Spinal Fractures, Stress, Mechanical, Survival Analysis, Survival analysis},
	pages = {90--98}
}

@article{devogel_application_2018,
	title = {Application of resampling techniques to improve the quality of survival analysis risk curves for human frontal bone fracture},
	issn = {1879-1271},
	doi = {10.1016/j.clinbiomech.2018.04.013},
	abstract = {BACKGROUND: In automotive events, head injuries (skull fractures and/or brain injuries) are associated with head contact loading. While the widely-used head injury criterion is based on frontal bone fracture and linear accelerations, injury risk curves were not developed from original datasets.
OBJECTIVES: Develop skull fracture-based risk curves for using previously published data and apply resampling techniques to assess their qualities.
METHODS: Force, deflection, energy, and stiffness data from thirteen human cadaver head impact tests were used to develop risk curves using parametric survival analysis. Injuries occurred to all specimens. Data points were treated as uncensored. Variables were ranked, and the variable best explaining the underlying fracture response was determined using the Brier Score Metric (BSM). The qualities of the risk curves were determined using normalized confidence interval sizes. Statistical resampling methods were used to assess the quality of the risk curves and the impact of the sample size by conducting 2000 simulations. Sample sizes ranged from 13 to 26.
FINDINGS: The Weibull distribution was optimal for all the response variables, except deflection (log-logistic). The quality of the risk curves was the highest for deflection. This variable best explained the underlying head injury response, based on BSM. Improvements in the quality of the risk curves were achieved with additional samples of force and deflection ({\textless}13), while energy and stiffness variables required more size. Individual risk curves are given.
INTERPRETATION: These probability curves from head contact loading add to the understanding skull fractures and can be used to improve safety in injury producing environments.},
	language = {eng},
	journal = {Clinical Biomechanics (Bristol, Avon)},
	author = {DeVogel, Nicholas and Banerjee, Anjishnu and Yoganandan, Narayan},
	month = apr,
	year = {2018},
	pmid = {29753560},
	keywords = {Biomechanics, Brier score metric, Head contact loading, Head injury, Skull fracture, Statistical resampling}
}

@article{yoganandan_novel_2018,
	title = {A {Novel} {Competing} {Risk} {Analysis} {Model} to {Determine} the {Role} of {Cervical} {Lordosis} in {Bony} and {Ligamentous} {Injuries}},
	volume = {119},
	issn = {1878-8769},
	doi = {10.1016/j.wneu.2018.08.011},
	abstract = {OBJECTIVE: To determine role of lordosis in cervical spine injuries using a novel competing risk analysis model.
METHODS: The first subgroup of published experiments (n = 20) subjected upright human cadaver head-neck specimens to impact loading. The natural lordosis was removed. The second (n = 21) and third (n = 10) subgroups of published tests subjected inverted specimens to head impact loading. Lordosis was preserved in these 2 subgroups. Using axial force and age as variables, competing risks analysis techniques were used to determine the role of lordosis in the risk of bone-only, ligament-only, and bone and ligament injuries.
RESULTS: Bony injuries were focused more at 1 level to a straightened spine, and ligament injuries were spread around multiple levels. Age was not a significant (P {\textless} 0.05) covariate. A straightened spine had 3.23 times higher risk of bony injuries than a lordotic spine. The spine with maintained lordosis had 1.14 times higher risk of ligament injuries, and 2.67 times higher risk of bone and ligament injuries than a spine without lordosis (i.e., preflexed column).
CONCLUSIONS: Increased risk of bony injuries in a preflexed spine and ligament injuries in a lordotic spine may have implications for military personnel, as continuous use of helmets in the line of duty affects the natural curvature; astronauts, as curvatures are less lordotic after missions; and civilian patients with spondylotic myelopathy who use head protective devices, as curvatures may change over time in addition to the natural aging process.},
	language = {eng},
	journal = {World Neurosurgery},
	author = {Yoganandan, Narayan and Banerjee, Anjishnu and DeVogel, Nicholas and Pintar, Frank A. and Baisden, Jamie L.},
	month = nov,
	year = {2018},
	pmid = {30114533},
	keywords = {Biomechanical Phenomena, Biomechanics, Cadaver, Cervical Vertebrae, Cervical spine, Competing risk analysis, Curvature, Humans, Impact loading, Injuries, Ligaments, Lordosis, Models, Statistical, Retrospective Studies, Risk Assessment, Sagittal alignment, Spinal Injuries, Statistical modeling},
	pages = {e962--e967}
}

@article{banerjee_novel_2018,
	title = {Novel learning framework (knockoff technique) to evaluate metric ranking algorithms to describe human response to injury},
	volume = {19},
	issn = {1538-957X},
	doi = {10.1080/15389588.2018.1519805},
	abstract = {OBJECTIVE: The objective of this study is to present a novel framework, termed the knockoff technique, to evaluate different metric ranking algorithms to better describe human response to injury.
METHODS: Many biomechanical metrics are routinely obtained from impact tests using postmortem human surrogates (PMHS) to develop injury risk curves (IRCs). The IRCs form the basis to evaluate human safety in crashworthiness environments. The biomechanical metrics should be chosen based on some measure of their predictive ability. Commonly used algorithms for the choice of ranking the metrics include (a) areas under the receiver operating characteristic curve (AUROC), time-varying AUROC, and other adaptations, and (b) some variants of predictive squared error loss. This article develops a rigorous framework to evaluate the metric selection/ranking algorithms. Actual experimental data are used due to the shortcoming of using simulated data. The knockoff data are meshed into existing experimental data using advanced statistical algorithms. Error rate measures such as false discovery rates (FDRs) and bias are calculated using the knockoff technique. Experimental data are used from previously published whole-body PMHS side impact sled tests. The experiments were conducted at different velocities, padding and rigid load wall conditions, and offsets and with different supplemental restraint systems. The PMHS specimens were subjected to a single lateral impact loading resulting in injury and noninjury outcomes.
RESULTS: A total of 25 metrics were used from 42 tests. The AUROC-type algorithms tended to have higher FDRs compared to the squared error loss-type functions (45.3\% for the best AUROC-type algorithms versus 31.4\% for the best Brier score algorithm). Standard errors for the Brier score algorithm also tended to be lower, indicative of more stable metric choices and robust rankings. The wide variations observed in the performance of the algorithms demonstrated the need for data set-specific evaluation tools such as the knockoff technique developed in this study.
CONCLUSIONS: In the present data set, the AUROCs and related binary classification algorithms led to inflated FDRs, rendering metric selection/ranking questionable. This is particularly true for data sets with a high proportion of censoring. Squared error loss-type algorithms (such as the Brier score algorithm or its modifications) improved the performance in the metric selection process. The presented new knockoff technique may wholly change how IRCs are developed from impact experiments or simulations. At the very least, the knockoff technique demonstrates the need for evaluations among different metric ranking/selection algorithms, especially when they produce substantially different biomechanical metric choices. Without recommending the AUROC-type or Brier score-type algorithms universally, the authors suggest careful assessments of these algorithms using the proposed framework, so that a robust algorithm may be chosen, with respect to the nature of the experimental data set. Though results are given for sets from a published series of experiments, the knockoff technique is being used by the authors in tests that are applicable to the automotive, aviation, military, and other environments.},
	language = {eng},
	number = {sup2},
	journal = {Traffic Injury Prevention},
	author = {Banerjee, Anjishnu and DeVogel, Nicholas and Pintar, Frank A. and Yoganandan, Narayan},
	year = {2018},
	pmid = {30570337},
	keywords = {Injury risk curves, bias measurement, data augmentation, false discovery rate, knockoff technique, ranking algorithms},
	pages = {S121--S126}
}

@article{devogel_hierarchical_2019,
	title = {Hierarchical process using {Brier} {Score} {Metrics} for lower leg injury risk curves in vertical impact},
	issn = {0035-8665},
	doi = {10.1136/jramc-2018-001124},
	abstract = {INTRODUCTION: Parametric survival models are used to develop injury risk curves (IRCs) from impact tests using postmortem human surrogates (PMHS). Through the consideration of different output variables, input parameters and censoring, different IRCs could be created. The purpose of this study was to demonstrate the feasibility of the Brier Score Metric (BSM) to determine the optimal IRCs and derive them from lower leg impact tests.
METHODS: Two series of tests of axial impacts to PMHS foot-ankle complex were used in the study. The first series used the metrics of force, time and rate, and covariates of age, posture, stature, device and presence of a boot. Also demonstrated were different censoring schemes: right and exact/uncensored (RC-UC) or right and uncensored/left (RC-UC-LC). The second series involved only one metric, force, and covariates age, sex and weight. It contained interval censored (IC) data demonstrating different censoring schemes: RC-IC-UC, RC-IC-LC and RC-IC-UC-LC.
RESULTS: For each test set combination, optimal IRCs were chosen based on metric-covariate combination that had the lowest BSM value. These optimal IRCs are shown along with 95\% CIs and other measures of interval quality. Forces were greater for UC than LC data sets, at the same risk levels (10\% used in North Atlantic Treaty Organisation (NATO)). All data and IRCs are presented.
CONCLUSIONS: This study demonstrates a novel approach to examining which metrics and covariates create the best parametric survival analysis-based IRCs to describe human tolerance, the first step in describing lower leg injury criteria under axial loading to the plantar surface of the foot.},
	language = {eng},
	journal = {Journal of the Royal Army Medical Corps},
	author = {DeVogel, Nicholas and Yoganandan, N. and Banerjee, A. and Pintar, F. A.},
	month = jan,
	year = {2019},
	pmid = {30709924},
	keywords = {fractures, impact biomechanics, injury, injury risk curves, survival analysis, trauma}
}